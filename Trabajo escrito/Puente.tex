\documentclass[16pt,a4papper]{article}
\usepackage[utf8]{inputenc}
\usepackage[T1]{fontenc}
\usepackage{amsmath}
\usepackage{amsfonts}
\usepackage{amssymb}
\usepackage{graphicx}
\usepackage[style=numeric]{biblatex}
\addbibresource{Referencias.bib}
\usepackage{xcolor}
\usepackage{listings}
\definecolor{mGreen}{rgb}{0,0.6,0}
\definecolor{mGray}{rgb}{0.5,0.5,0.5}
\definecolor{mPurple}{rgb}{0.58,0,0.82}
\definecolor{backgroundColour}{rgb}{0.95,0.95,0.92}

\lstdefinestyle{CStyle}{
	backgroundcolor=\color{backgroundColour},   
	commentstyle=\color{mGreen},
	keywordstyle=\color{magenta},
	numberstyle=\tiny\color{mGray},
	stringstyle=\color{mPurple},
	basicstyle=\footnotesize,
	breakatwhitespace=false,         
	breaklines=true,                 
	captionpos=b,                    
	keepspaces=true,                 
	numbers=left,                    
	numbersep=5pt,                  
	showspaces=false,                
	showstringspaces=false,
	showtabs=false,                  
	tabsize=2,
	language=C
}

\title{Proyecto 1: Puente}
\author{José David Flores Rodríguez}
\date{\today}

\begin{document}
	\begin{center}
		
		\includegraphics[width=200px]{Logo-UNA.png}
		\vspace{1cm}

	
		\large
		Universidad Nacional de Costa Rica
		\vspace{1cm}
		
	
		\large
		Escuela de informática
		\vspace{1cm}
		

		\large
		Proyecto de Sistemas Operativos: Puente
		\vspace{1cm}
		

		\large
		Estudiante: José David Flores Rodríguez
		\vspace{1cm}
		
	
		\large
		Profesor: Eddy Rampirez
		\vspace{1cm}
	
		\large
		Fecha \today
		\vspace{1cm}
	\end{center}
	
	\newpage
	\section{Introducción}
	Para este proyecto se nos pide realizar una simulación de un puente el cual tiene solo una vía, o sea solo un sentido. Esta vía o sentido puede ser de: este a oeste o de oeste a este, dependiendo de ciertos factores como lo son: semáforos vehiculares, policías de transito o el que "llega de primero". Dichas entidades serán las encargadas de controlar el paso por el puente para evitar choques entre los vehículos y aprovechar al máximo este recurso (el puente).\par
	
	Para realizar esta simulación debíamos utilizar el lenguaje de programación C y el sistema operativo linux. Además de usar los hilos de la librería POSIX para la implementación de las entidades (vehículos, semáforos, oficiales de transito, ambulancias...).\par

	En este texto se explicarán las librerías de C utilizadas para la elaboración del proyecto, la descripción de la implementación del problema propuesto, descripción de la experiencia durante la realización del mismo y aprendizajes.\par
	
	\newpage
	\section{Marco Teórico}
	Sin duda alguna en un programa los hilos son de gran importancia ya que permite la ejecución de varias tareas de manera "simultanea", sin embargo, aunque se perciba que las tareas se realizan al mismo tiempo, esto no es así; cada núcleo del procesador puede realizar una tarea a la vez, lo que sucede es que este trabaja a una velocidad tan rápida que nuestra percepción es que se ejecutan al mismo tiempo, más aún si el procesador es multi núcleo lo que si permitiría la ejecución de varios procesos de forma paralela(al mismo tiempo). En el lenguaje de programación C existe una librería muy popular para la administración de hilos la cual esta basada en POSIX. POSIX es un una interfaz "standart" para todos los sistemas operativos la cual permite que el programa sea portable y pueda ser ejecutado en diferentes sistemas operativos basados en POSIX, según \cite{POSIX_documentation}.\par 
	
	La librería de la cual se habla en el párrafo anterior se llama Pthread y esta tiene las siguientes funciones que permiten la administración de hilos:\par
	
	\begin{itemize}
		\item \textbf{pthread\_t:} Usado para identificar un hilo.
		\item \textbf{pthread\_create(pthread\_t *thread, const pthread\_attr\_t *attr,
			void *(*start\_routine)(void*), void *arg):} Permite crear un nuevo hilo. Los atributos que se desean utilizar para el hilo se especifican en el parámetro "attr", el parámetro "start\_routine" es la función que se va a ejecutar y el parámetro "arg" recibe los argumentos que se quieran pasar a la función que se va a ejecutar.
		\item \textbf{pthread\_exit(void *value\_ptr):} Esta función termina la ejecución del hilo que la llama, permite que el atributo referenciado en "value\_ptr" sea accedido cuando se utilice la función "pthread\_join".
		\item \textbf{pthread\_join(pthread\_t thread, void **value\_ptr):} Permite suspender la ejecución del hilo que lo llama hasta que el hilo referenciado en el parámetro "thread" haya terminado su ejecución. El valor de "value\_ptr" contiene el valor que pasó desde la función "pthread\_exit".
		\item \textbf{int pthread\_mutex\_init(pthread\_mutex\_t *mutex, const pthread\_mutexattr\_t *attr):} Función que permite que se inicie un MUTEX con los atributos especificados en el parámetro "atr".
		\item \textbf{int pthread\_mutex\_lock(pthread\_mutex\_t *mutex):} El MUTEX se bloquea cuando esta función es llamada.
		\item \textbf{int pthread\_mutex\_trylock(pthread\_mutex\_t *mutex):} Esta función intentará bloquear el MUTEX y retornará el entero 0 si la operación fue exitosa.
		\item \textbf{int pthread\_mutex\_unlock(pthread\_mutex\_t *mutex):} El MUTEX se desbloquea cuando se llama pthread\_mutex\_unlock.	
	\end{itemize}

	Las funciones anteriores son solo algunas de disponibles en la librería pthread, al igual que estas, la totalidad de funciones se pueden consultar en la documentación de "pthread.h" \cite{Pthread_documentation}. 

	\newpage
	\section{Descripción de la solución}
	
	\subsection{Semáforos}
	Para implementar el control del tránsito mediante semáforos se utilizaron las siguientes estructuras:
	
	\begin{lstlisting}[style=CStyle]
		struct semaforo
		{
			int tiempo_verde_este;
			int tiempo_verde_oeste;
			int este; //este = 1 si luz verde de este esta encendido, este = 0 si no 
			int oeste; //oeste = 1 si luz verde de oeste esta encendido, oeste = 0 si no
			int uso_timer; //Identifica si hay algun semaforo utilizando el timer para pasar de verde a rojo
			pthread_mutex_t mutex; 
			struct puente *puente; 
		};
		
		struct puente
		{
			int sentido;        //sentido = 1 si los automoviles van al este y sentido = 0 si van al oeste
			int sentido_actual; //sentido = 1 si los automoviles van al este y sentido = 0 si van al oeste
			int longitud_puente;
			pthread_mutex_t *puente_lock; //Arreglo de MUTEX el cual representa el puente
		};
		
		struct info_autos 
		{
			int velocidad_auto_este;
			int velocidad_auto_oeste;
			int media_este;
			int media_oeste;
			struct semaforo *semaforos;
			struct puente *puente;
		};
	\end{lstlisting}
	
	La estructura que se pasa como argumento en la mayoría de los hilos es "info\_autos" la cual contiene toda la información que los automóviles necesitan para decidir si pasar o no.
	
	Las funciones utilizadas en la implementación fueron las siguientes:
	
	\begin{lstlisting}[style=CStyle]
		int iniciar_semaforo(); //Funcion principal que se encarga de iniciar las variables
		int main_runner(struct info_autos *info); //Esta funcion se encarga de crear y ejecutar threads que iniciaran la creacion de los semaforos y automoviles
		void *create_semaforos(void *arg);        //Esta funcion encapsula la creacion del semaforo ubicado al este y oeste del puente
		void *cambiar_semaforo_este(void *arg);   //Enciende el semaforo del oeste
		void *cambiar_semaforo_oeste(void *arg);  //Enciende el semaforo del oeste
		void *create_automoviles(void *arg);      //Funcion que crea los automoviles
		void *automovil_este(void *arg);          //Crea los automoviles por que entran por el este
		void *automovil_oeste(void *arg);         //Crea los automoviles que entran por el oeste
		void *crear_autos_este(void *arg);		  //Automovil que entra por el este	
		void *crear_autos_oeste(void *arg);       //Automovil que entra por el oeste
		double ejecutar_integral(struct info_autos *info, char eleccion_media[]); //Funcion que ejecuta la integral exponencial encargada de decidir la velocidad con que se crean los automoviles;
		int revisar_puente_en_uso(struct puente *puente);                         //funcion que recorre el arreglo de mutex revisando si hay un automovil usado el puente o no
	\end{lstlisting}

	\subsubsection{Semáforos}
	Los semáforos están siempre en un loop infinito verificando si el semáforo contrario cerro el paso y encendió la luz roja, cuando el semáforo contrario apagó la luz verde, el semáforo actual enciende la luz verde e inicia el timer mediante la función sleep().
	
	\subsubsection{Puente}
	El puente mediante las variables sentido y sentido\_actual permiten a los automóviles saber si tienen derecho de pasar o no. El sentido tomando en cuenta el estado del semáforo, sin embargo, sentido\_actual cambia hasta que el puente este desocupado de automóviles que ya estaban en el puente cunado el semáforo cambio de estado.
	
	\subsubsection{Automóviles}
	Estos toman la decisión de pasar si el sentido y sentido\_actual son iguales, sino estos no pasaran por el puente.

	\newpage
	\printbibliography
	
	
\end{document}